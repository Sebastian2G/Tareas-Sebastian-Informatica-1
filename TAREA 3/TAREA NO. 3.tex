\documentclass{article}

\usepackage{amsmath}
\usepackage{amssymb}
\usepackage{graphicx}
\topmargin=-1in
\evensidemargin=0in
\oddsidemargin=0in
\textwidth=6.5in
\textheight=9.0in
\headsep=0.25in

\begin{document}
\title{\textbf {Tarea No. 3}}
\author{Sebastian Gomez}
\maketitle
\newline
\begin{center}
\textbf EJERCICIO NO. 1
\end{center}
\begin{itemize}
\item 3 + 4
\item ssso + sssso
\item s(ssso + ssso)
\item s(s(ssso + sso))
\item s(s(s(ssso + so)))
\item s(s(s(s(ssso+o))))
\item s(s(s(s(s(sso+o)))))
\item s(s(s(s(s(s(so+o)))))) 
\item Por definición a + 0 = 0 entonces, 
\item s(s(s(s(s(s(so+o)))))) = ssssssso
\item ssssssso = 7
\end{itemize}
\newline
\begin{center}
\textbf EJERCICIO NO. 2
\end{center}

\begin{itemize}
\item Caso base: a = 0
\item 0 x b = 0.
\item Esto se refiere a que cualquier número por cero es igual a cero.
\item Caso Inductivo 
\item a x b = a + (a x (b - 1))
\item a x b = a + ab - a 
\item a x b = ab
\item a x b = a x b
\end{itemize}
\newline
\begin{center}
\textbf EJERCICIO NO. 3
\end{center}
\begin{itemize}
\item s(s(s(o))) x 0
\item En el ejercicio anterior se definio que cualquier número por 0 es 0. 
\newline
\item s(s(s(o))) x s(o) 
\item s(s(s(o))) x s(o) = ssso + (ssso x (so - so)
\item s(s(s(o))) x s(o) = ssso + (ssso x (0)
\item s(s(s(o))) x s(o) = ssso
\item 3 x 1 = 3
\newline
\item s(s(s(o))) x s(s(o))
\item s(s(s(o))) x s(s(o)) = ssso + (ssso x (sso - so)
\item s(s(s(o))) x s(s(o)) = ssso + (ssso x so)
\item s(s(s(o))) x s(s(o)) = ssso + ssso
\item s(s(s(o))) x s(s(o)) = sssssso
\item 3 x 2 = 6
\end{itemize}
\newline
\begin{center}
\textbf EJERCICIO NO. 4
\end{center}

\begin{itemize}
\newline
\item a + s(s(o)) = s(s(a))
\item a + s(s(o)) = s(o) + s(a)
\item a + s(s(o)) = s(s(o)) + a
\item a + s(s(o)) = a + s(s(o))
\newline
\newline
\item a x b = b x a
\item s(a) x s(b) = s(b) x s(a)
\item s(a x b) = s(b x a)
\item s(a x b)/s = s(b x a)/s
\item a x b = b x a
\newline
\newline
\item Caso base c = 0
\item a x ( b x 0) = (a x b) x 0
\item a x 0 = ab x 0
\item 0 = 0 
\newline
\item Caso Inductivo 
\item a x ( b x c) = (a x b) x c
\item a x bc = ab x c
\item abc = abc
\newline
\newline
\item Caso base c = 0
\item (a + b) x 0 = (a x 0) + (b x 0)
\item 0 = 0 + 0
\item 0 = 0
\newline
\item Caso Inductivo
\item (a + b) x c = (a x c) + (b x c)
\item ac + bc = ac + bc


\end{itemize}




\end{document}