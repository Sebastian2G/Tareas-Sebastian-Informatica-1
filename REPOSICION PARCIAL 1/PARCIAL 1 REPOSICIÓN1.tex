\documentclass{article}
\usepackage{amsmath}
\usepackage{amssymb}
\usepackage{graphicx}
\topmargin=-1in
\evensidemargin=0in
\oddsidemargin=0in
\textwidth=6.5in
\textheight=9.0in
\headsep=0.25in

\begin{document}
\title{\textbf {RECUPERACIÓN PARCIAL 1}}
\author{Sebastian Gomez}
\maketitle

\begin{center}
\textbf{PREGUNTA #1}
\end{center}

\begin{itemize}
\item El conjunto de Nodos es 1, 2, 3, 4, 5, 6, 7
\item $<1,2>$ $<1,3>$ $<1,4>$ $<1,5>$ $<1,6>$ $<2,3>$ $<2,4>$ $<2,5>$ $<2,6>$ $<3,4>$ $<3,5>$ $<3,6>$ $<3,7>$ $<4,7>$ $<5,6>$ $<5,7>$ $<6,7>$
\end{itemize}

\begin{center}
\textbf{PREGUNTA #2}
\end{center}

\[
\sum_{i=1}^{n}i= \frac{n (n + 1)}{2}
\]

\begin{itemize}
\item Caso Base
\item n = 1
\item \[
\sum_{i=1}^{1}i= \frac{1 (1 + 1)}{2}
\]
\item \[
\sum_{i=1}^{1}i= 1
\]

\item Caso Inductivo
\item \[
\sum_{i=1}^{n}i= \frac{n (n + 1)}{2}
\]

\item n = k
\item \[
\sum_{i=1}^{k}i= \frac{k (k + 1)}{2}
\]

\item n = k + 1

\item \[
\sum_{i=1}^{k+1}i= \frac{(k + 1) ((k + 1) + 1)}{2}
\]

\item \[
\sum_{i=1}^{k}i + (k + 1) = \frac{(k + 1)(k + 2)}{2}
\]
\item \[ \frac{k (k + 1) + 2(k + 1)}{2} = \frac{(k + 1)(k + 2)}{2}
\]

\item  \[ \frac{k^2 + k + 2k + 2}{2} = \frac{k^2 + k + 2k + 2}{2}
\]
\item Queda demostrado al ser igual.
\end{itemize}

\begin{center}
\textbf{PREGUNTA #3}
\end{center}

\begin{itemize}
\item Para todo numero a + b si a = 0 entonces a + b = b. Y si a = s(i) entonces a + b = s (i + b)
\item Entoces para la Sumatoria 
\end{itemize}

\begin{center}
\textbf{PREGUNTA #4}
\end{center}

\begin{itemize}
\item a = s(0)
\item a = s(s(0))
\item a + b = b + a
\item s(0) + s(s(0)) = s(s(0)) + s(0)
\item s(s(0) + s(0)) = s(s(s(0) + 0
\item s(s(s(0) + 0 = s(s(s(0) + 0
\item s(s(s(0) = s(s(s(0)
\item Esto quiere decir que ambos son iguales.
\end{itemize}

\begin{center}
\textbf{PREGUNTA #5}
\end{center}
\begin{itemize}
\item $((n + n) \geq n ) = s(0)$
\item $s(0) + s(0) \geq s(0)$
\item $s(s(0)) \geq s(0)$
\item $s(s(0)) - s(0) \geq 0$
\item $s(0) \geq 0$
\end{itemize}



\end{document}
